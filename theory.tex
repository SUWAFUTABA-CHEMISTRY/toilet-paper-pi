\documentclass[a4j]{jarticle}
\usepackage{amsmath, amssymb}
\usepackage[dvipdfmx]{graphicx}
\usepackage[margin=10mm]{geometry}
\usepackage{float}
\usepackage{multicol}
\numberwithin{equation}{section}
\title{トイレットペーパーを用いた円周率の求め方に関する考察}
\author{部長 \thanks{諏訪二葉高校化学部(SFCC)}}
\begin{document}
\maketitle
\section{Intruduction}
 円周率の計算方法を考えていた時に思いついたのでここに記す。
\section{Methods}
  紙が$N$枚巻かれているトイレットペーパー1ロールを考える。ここで紙の厚さを$\{r_n\}(n=2, 3, 4, \cdots)$で表す。また芯の半径を$r_1$とする。ここで$n$周目に1周,紙を巻いた時,その紙の長さを$l_n$とする。この時,$l_n$は以下の式で表される。
\begin{equation*}
l_n = 2 \pi \left(\sum^n_k r_k\right)
\end{equation*}
ここで$\{l_n\}$の階差数列$\{l'_n\}$の漸化式を考えると,その式は以下のようになる。
\begin{equation*}
l'_n = l_{n+1} - l_n (n=1, 2, \cdots N-1)
\end{equation*}
また$l_n$の定義を思い出すと,以下のようにも表される。
\begin{equation*}
l'_n = 2\pi\left\{\sum^{n+1}_{k_1} r_{k_1} -  \sum^{n}_{k_2} r_{k_2} \right\} = 2\pi r_{n+1}
\end{equation*}
 従って,円周率$\pi$は以下の式で表される。
\begin{equation*}
\pi = \frac{l_{n+1} - l_n}{2r_{n+1}}
\end{equation*}
 精度を上げるため,この式をトイレットペーパー全体に適用する。それを考慮して,以下のような式を定める。
\begin{equation*}
\pi = \frac{1}{N} \sum^N_n \frac{l_{n+1} - l_n}{2r_{n+1}}
\end{equation*}
\section{Discussion}
 この方法は,$l$の値と$r$の値を計測さえすれば円周率が計算できるものとなっている。検証は後ほど行うものとする。
\end{document}